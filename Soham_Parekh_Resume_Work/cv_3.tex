%%%%%%%%%%%%%%%%%%%%%%%%%%%%%%%%%%%%%%%%%
% Freeman Curriculum Vitae
% XeLaTeX Template
% Version 2.0 (19/3/2018)
%
% This template originates from:
% http://www.LaTeXTemplates.com
%
% Authors:
% Vel (vel@LaTeXTemplates.com)
% Alessandro Plasmati
%
% License:
% CC BY-NC-SA 3.0 (http://creativecommons.org/licenses/by-nc-sa/3.0/)
%
%!TEX program = xelatex
% NOTICE: This template must be compiled with XeLaTeX, the line above should
% ensure this happens automatically but if it doesn't you will need to specify 
% XeLaTeX as the engine in your editor or script
% 
%%%%%%%%%%%%%%%%%%%%%%%%%%%%%%%%%%%%%%%%%

%----------------------------------------------------------------------------------------
%	PACKAGES AND OTHER DOCUMENT CONFIGURATIONS
%----------------------------------------------------------------------------------------

\documentclass[10pt]{article} % Font size, can be: 10pt, 11pt or 12pt

\input{structure.tex} % Include the file that specifies the document structure

% Headers and footers can be added with the \lhead{} \rhead{} \lfoot{} \rfoot{} commands
% Example right footer:
%\rfoot{\color{headings}{\sffamily Last update: \today. Typeset with Xe\LaTeX}}

%----------------------------------------------------------------------------------------

\begin{document}

\begin{paracol}{2} % Begin the multi-column environment

%----------------------------------------------------------------------------------------
%	NAME AND CURRICULUM VITAE TEXT
%----------------------------------------------------------------------------------------

\parbox[top][0.10\textheight][c]{\linewidth}{ % Parbox to hold the author name and CV text; fixed height to match the coloured box to the right, centred vertically and full line width
	\vspace{-0.04\textheight} % Reduce whitespace above the parbox to separate it from the main content
	\centering % Centre text
	{\sffamily\Huge Soham Parekh}\\\medskip % Your name
	{\Huge\color{headings} Software Developer}
}

%----------------------------------------------------------------------------------------
%	WORK EXPERIENCE
%----------------------------------------------------------------------------------------

\section{Relevant Experience}

% Blank \workposition command to add another job:

%\workposition{} % Duration
%{} % FT/PT (full time or part time)
%{}
%{} % Employer | Job Title
%{} % Description

% All 5 parameters must be supplied but any can be empty if you don't need them

%------------------------------------------------

\workposition{Current, from Sept 2020} % Duration
{FT} % FT/PT (full time or part time)
{} 
{Betterbank.app | \small{SDE Intern}} % Employer | Job Title
{
\textbullet{} Designed a performant backend GraphQL server. \\
\textbullet{} Introduced IP-based rate limiting and other mechanisms to improve overall security of the app against DDoS and MITM attacks, CSRF and XSS. \\
\textbullet{} Implemented a distributed queuing system using producer-consumer model to offload heavy background jobs. \\
\textbullet{} Refactored and improved existing test coverage by 31\%
} % Description

%------------------------------------------------

\workposition{May 2020 -- Sept 2020} % Duration
{FT} % FT/PT (full time or part time)
{}
{Wikimedia Foundation | \small{Google Summer of Code Developer}} % Job title
{
\textbullet{} Investigated Puppeteer with Jest and Cypress. \\
\textbullet{} Introduced an out-of-the-box solution for recording videos of Puppeteer tests. \\
\textbullet{} Designed a custom Jest preset to spin up environment for Mediawiki tests.
}  % Description

%------------------------------------------------

\workposition{Dec 2019 -- March 2020} % Duration
{FT} % FT/PT (full time or part time)
{} % Employer
{Falkonry | \small{SDE Intern}} % Job title
{
\textbullet{} Redesigned and programmed their CI/CD bot to work with multiple GitHub repositories and custom configurations. \\
\textbullet{} Developed a mechanism to create Kubernetes deployments in a completely air-gapped hybrid environment. \\
\textbullet{} Devised custom methods to efficiently query and parse environment logs, reducing the computation time by 34.83\%.
} % Description

%------------------------------------------------

%----------------------------------------------------------------------------------------
%	SELECTED PROJECTS
%----------------------------------------------------------------------------------------

\section{Selected Projects}

\workposition{Ongoing, May 2019} % Duration
{}
{}
{Hydrabot: Automation Bot} % Project title
{
\textbullet{} Programmed a configurable  automation bot for GitOps workflows. \\
\textbullet{} Integrated issue auto-labelling using unsupervised Graph-based Topic Labeling. \\
\textbullet{} Orchestrated a highly available cluster capable of handing 5000 req/minute.
} % Description

\workposition{Jan 2020 -- Sept 2020} % Duration
{}
{}
{ApplyByAI: Recruitment Platform} % Project title
{
\textbullet{} Proposed a novel Recruitment System based on Big5 personality analysis, skill matching and large scale hybrid filtering. \\
\textbullet{} Utilized contextual information using tone and sentiment analysis of user responses. \\
\textbullet{} Conserved 22.3\% time spent during college committee interviews.
} % Description

\workposition{May 2019 -- April 2020} % Duration
{}
{}
{Friendly: Social Network} % Project title
{
\textbullet{} Ideated \& led research on a novel Friend Recommendation System using Big5 personality analysis \& hybrid filtering. \\
\textbullet{} Outperformed traditional systems with an average personality match of 83.27\%.
} % Description

\workposition{Feb 2019 -- Sept 2019} % Duration
{}
{}
{Human Detection in post-disaster LWIR Imagery} % Project title
{
\textbullet{} Proposed a novel algorithm to detect humans in low resolution infrared-imagery. \\
\textbullet{} Leveraged Maximally Stable Extremal Hotspots as key points for edge detection. \\
\textbullet{} Trained a Bayesian Classifier on hot spots processed using Discrete cosine transform as a descriptor. \\
\textbullet{} Achieved  87.527\% accuracy on OTCBVS Dataset.
} % Description

\vspace{-\baselineskip}\medskip % Standardise the whitespace after this section and before the next (the custom command adds too much otherwise)

\switchcolumn % Switch to the next paracol column

%----------------------------------------------------------------------------------------
%	COLOURED CONTACT DETAILS BOX
%----------------------------------------------------------------------------------------

\parbox[top][0.10\textheight][c]{\linewidth}{ % Parbox to hold the colour box; fixed height to match the name/CV text to the left, centred vertically and full line width
	\vspace{-0.04\textheight} % Reduce whitespace above the parbox to separate it from the main content
	\colorbox{shade}{ % Create the coloured box
		\begin{supertabular}{p{0.05\linewidth}|p{0.775\linewidth}} % Start a table with two columns, the table will ensure everything is aligned
			\raisebox{-1pt}{\faHome} & Mumbai, India \\ % Address
			\raisebox{0pt}{\small\faEnvelope} & \href{mailto:soham.parekh1998@gmail.com}{soham.parekh1998@gmail.com} \\ % Email address
			\raisebox{-1pt}{\small\faDesktop} & \href{https://sohamp.dev}{https://sohamp.dev} \\ % Website
			\raisebox{-1pt}{\faGithub} & \href{https://github.com/und3fined-v01d}{https://github.com/und3fined-v01d} \\ % GitHub profile
			\raisebox{-1pt}{\faLinkedinSquare} & \href{https://linkedin.com/in/soham-parekh}{https://linkedin.com/in/soham-parekh} \\ % LinkedIn profile
			% See fontawesome.pdf in the fonts folder for all icons you can use
		\end{supertabular}
	}
}

%----------------------------------------------------------------------------------------
%	EDUCATION
%----------------------------------------------------------------------------------------

\section{Education} 

% Blank \educationentry{} command to add another degree:

%\educationentry{} % Duration
%{} % Degree
%{} % Honours, achievements or distinctions (e.g. first class honours)
%{} % Department
%{} % Institution

% All 5 parameters must be supplied but any can be empty if you don't need them

%------------------------------------------------

\begin{supertabular}{rl} % Start a table with two columns, the table will ensure everything is aligned

	%------------------------------------------------
	
	\educationentry{2017 -- 2021} % Duration
	{B.E. Computer Engineering} % Degree
	{Cum. GPA: 9.78/10} % Honours, achievements or distinctions (e.g. first class honours)
	{D.J. Sanghvi College of Engineering} % Department
	{University of Mumbai} % Institution
	
	%------------------------------------------------

\end{supertabular}

%----------------------------------------------------------------------------------------
%	SKILLS
%----------------------------------------------------------------------------------------

\section{Skills} 

% Example \tableentry{} command to add another line:

%\tableentry{Heading}{Content}{spaceafter}

% All 3 parameters must be supplied but any can be empty if you don't need them
% A "spaceafter" value in the third parameter will add some vertical space -- this is to be used between headings

%------------------------------------------------

\begin{supertabular}{rl} % Start a table with two columns, the table will ensure everything is aligned
	
	%------------------------------------------------
	
	\tableentry{Programming}{\textit{over 5000 lines:}}{}
	\tableentry{}{C/C++, Javascript, Python, Go,}{}
	\tableentry{}{\textit{over 1000 lines:}}{}
	\tableentry{}{Rust, Java, Bash, Haskell}{spaceafter}
	
	%------------------------------------------------
	
	\tableentry{Web}{React, Vue.js, Node.js, Django,}{}
	\tableentry{Development}{Flask, GraphQL, Rest}{spaceafter}
	
	%------------------------------------------------
	
	\tableentry{Devops \&}{Jenkins, Docker, Kubernetes,}{}
	\tableentry{Testing}{Ansible, ELK Stack, Jest, Cypress,}{}
	\tableentry{}{Puppeteer, WebdriverIO}{spaceafter}
	
	%------------------------------------------------
	
	%------------------------------------------------
	
	\tableentry{Libraries \&}{Numpy, Pandas, NLTK, Pytorch,}{}
	\tableentry{Frameworks}{Scikit-learn, Tensorflow, Electron}{spaceafter}
	
	%------------------------------------------------
	
\end{supertabular}

%----------------------------------------------------------------------------------------
%	AWARDS
%----------------------------------------------------------------------------------------

\section{Awards}

% Example \tableentry{} command to add another line:

%\tableentry{Heading}{Content}{spaceafter}

% All 3 parameters must be supplied but any can be empty if you don't need them
% A "spaceafter" value in the third parameter will add some vertical space -- this is to be used between headings

%------------------------------------------------

\begin{supertabular}{rl} % Start a table with two columns, the table will ensure everything is aligned
	
	%------------------------------------------------
	
	\tableentry{2020}{\textbf{Winner -- HackerEarth Recruit-a-thon}}{}
	\tableentry{}{\textit{among 2428 participating teams}}{spaceafter}
	
	%------------------------------------------------
	
	\tableentry{2020}{\textbf{Winner -- JP Morgan Code for Good}}{}
	\tableentry{}{\textit{among 80 participating teams}}{spaceafter}
	
	%------------------------------------------------
	
	\tableentry{2020}{\textbf{Top 20 -- IBM Hack Challenge}}{}
	\tableentry{}{\textit{among 800 participating teams}}{spaceafter}
	
	%------------------------------------------------
	
\end{supertabular}

%----------------------------------------------------------------------------------------
%	LEADERSHIP
%----------------------------------------------------------------------------------------

\section{Leadership}

% Example \tableentry{} command to add another line:

%\tableentry{Heading}{Content}{spaceafter}

% All 3 parameters must be supplied but any can be empty if you don't need them
% A "spaceafter" value in the third parameter will add some vertical space -- this is to be used between headings

%------------------------------------------------

\begin{supertabular}{rl} % Start a table with two columns, the table will ensure everything is aligned
	
	%------------------------------------------------
	\tableentry{Mentor}{Script Winter of Code}{}
	\tableentry{(Dec 2020 - Present)}{}{spaceafter}
	
	%------------------------------------------------
	
	\tableentry{Outreachy Mentor}{Wikimedia Foundation}{}
	\tableentry{(Oct 2020 - Present)}{}{spaceafter}
	
	%------------------------------------------------
	
	\tableentry{GSoC Co-mentor}{Cloud Native Computing}{}
	\tableentry{(May - Sept 2020)}{Foundation}{spaceafter}
	
	%------------------------------------------------
	
	\tableentry{Placement Coordinator}{DJ Sanghvi Placement}{}
	\tableentry{(Jan - Nov 2020)}{Department}{spaceafter}
	
	%------------------------------------------------
	
\end{supertabular}

%----------------------------------------------------------------------------------------
%	Volunteer Work
%----------------------------------------------------------------------------------------

\section{Volunteer Work}

% Example \tableentry{} command to add another line:

%\tableentry{Heading}{Content}{spaceafter}

% All 3 parameters must be supplied but any can be empty if you don't need them
% A "spaceafter" value in the third parameter will add some vertical space -- this is to be used between headings

%------------------------------------------------

\begin{supertabular}{rl} % Start a table with two columns, the table will ensure everything is aligned
	
	%------------------------------------------------
	
	\tableentry{Open Source Developer}{Wikimedia Foundation}{spaceafter}
	
	%------------------------------------------------
	
% 	\tableentry{Open Source Developer}{Cypress.io}{spaceafter}
	
	%------------------------------------------------
	
	\tableentry{Open Source Developer}{Jest (Facebook)}{spaceafter}
	
	%------------------------------------------------
	
	\tableentry{Open Source Developer}{Taskforce.sh}{spaceafter}
	
	%------------------------------------------------
	
\end{supertabular}

%----------------------------------------------------------------------------------------
%	CERTIFICATIONS
%----------------------------------------------------------------------------------------

\section{Certifications}

\begin{enumerate}
    \item Machine Learning with Python -- IBM (2020)
    \item Data Analysis using Python -- IBM (2020)
    \item Cloud Core -- IBM (2020)
\end{enumerate}

%----------------------------------------------------------------------------------------

\end{paracol}

%----------------------------------------------------------------------------------------

\end{document}
